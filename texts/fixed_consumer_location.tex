\documentclass{article}
\usepackage[utf8]{inputenc}
\usepackage[allcolors=blue]{hyperref}
\usepackage[left=2.3cm, right=2.3cm, scale=0.75,top=2cm, bottom=2cm]{geometry}
\usepackage[all]{hypcap}
\usepackage{caption}
\usepackage{natbib} % Use natbib for citing
\usepackage{amsfonts}
\usepackage{amsmath}
\usepackage{graphicx} 
\usepackage{tikz}
\graphicspath{{../graphs/}}
\usetikzlibrary{patterns}




\title{Fixed consumer location}
\author{Iana Gerina}
\date{\today}
%\geometry{
% a4paper,
% left=20mm,
% top=15mm,
% right=20mm,
% bottom=15mm,
% }

\begin{document}

\maketitle


Assume now that location of consumer is fixed and in the second period, partially revealed via his choice of the product. Indeed, if the consumer chooses the entrant, in second period both producers know, that:
$p_1^{inc}+t*x>p_1^{e}+t(y-x)$. And since the prices of the first period are revealed by the second period and t is constant, the location x becomes clearer to both producers. It makes the second period interaction a bit more interesting, as now both the producers have additional information. 

We'll start again with the second period. Suppose \textbf{the entrant was chosen}. Then we know that $p_1^{inc}+t*x>p_1^{e}+t(y-x)$. Let us denote $ \frac{p_2^{e}-p_2^{inc}}{2t}+1/2$ as f and $\frac{p_1^{e}-p_1^{inc}}{2t}+y/2$ as e. For the incumbent the probability of selling the good to the consumer in second period, if entrant is noticed, equals: $Pr(x<\frac{p_2^{e}-p_2^{inc}}{2t}+1/2 | x > \frac{p_1^{e}-p_1^{inc}}{2t}+y/2) = Pr(x<f|x>e)$, and for the entrant it is equal: $ Pr(x>\frac{p_2^{e}-p_2^{inc}}{2t}+1/2 | x >\frac{p_1^{e}-p_1^{inc}}{2t}+y/2 = Pr(x>f|x>e)$.
Evidently, the first period interaction results in two separate outcomes for different values of f and e. Suppose this condition holds:

$$ \frac{p_2^{e}-p_2^{inc}}{2t}+1/2 > \frac{p_1^{e}-p_1^{inc}}{2t}+y/2,$$
meaning \pmb{$f>e$}. We denote with red colour the condition on e and with blue the condition on f. 

%new ind cons to the left.
\begin{tikzpicture}
    % entrant 
    
    \draw[step=0.4cm,gray,very thin] (-2,-1) grid (3.6,2) node[anchor=north east, black] {entrant};
    \draw[thick] (-1.6,0) -- (2.4,0);
    \draw[thick, ->] (-1.6,0) -- (-1.6,1.6);
    \draw (-1.55,0.8) -- (-1.65,0.8) node[anchor=east] {1};
    \draw[thick, dashed] (2.4,0) -- (3.2,0);
    \draw (-1.6,1pt) -- (-1.6,-1pt) node[anchor=north] {0};
    \draw (2.4,1pt) -- (2.4,-1pt) node[anchor=north] {1};
    \draw (3.2,1pt) -- (3.2,-1pt) node[anchor=north] {y};
    \draw (1.2,1pt) -- (1.2,-1pt) node[anchor=north] {f};
    \draw[label distance=10mm] (0,1pt) -- (0,-1pt) node[anchor=north] {e};
    \draw[pattern=north east lines, pattern color=red] (0,0) rectangle (2.4,0.8);
    \draw[pattern=north west lines, pattern color=blue] (1.2,0) rectangle (2.4,0.8);

    % incumbent
    
    \draw[step=0.4cm,gray,very thin] (6.4,-1) grid (12,2) node[anchor=north east, black] {incumbent};
    \draw[thick] (6.8,0) -- (10.8,0);
    \draw[thick, dashed] (10.8,0) -- (11.6,0);
    \draw[thick, ->] (6.8,0) -- (6.8,1.6);
    \draw (6.75,0.8) -- (6.85,0.8) node[anchor=east] {1};
    \draw (6.8,1pt) -- (6.8,-1pt) node[anchor=north] {0};
    \draw (10.8,1pt) -- (10.8,-1pt) node[anchor=north] {1};
    \draw (11.6,1pt) -- (11.6,-1pt) node[anchor=north] {y};
    \draw (8.4,1pt) -- (8.4,-1pt) node[anchor=north] {e};
    \draw (9.6,1pt) -- (9.6,-1pt) node[anchor=north] {f};
    \draw[pattern=north east lines, pattern color=blue] (6.8,0) rectangle (9.6,0.8);
    \draw[pattern=north west lines, pattern color=red] (8.4,0) rectangle (10.8,0.8);
    

\end{tikzpicture}

From the graph we see that the demand the entrant faces is equal to $Pr(x>f|x>e)=\frac{1-f}{1-e}$. Demand of the incumbent, on the other hand, is is equal to $Pr(x \leq f| x >e) = \frac{f-e}{1-e}$. Substituting these equations into the corresponding profit functions and solving for optimal price, we derive the following equilibrium prices:
$$ p_2^{inc} = t+c+ 2/3(p_1^{inc}-p_1^{e}-yt)$$
$$ p_2^{e} = t+c + 1/3(p_1^{inc}-p_1^{e}-yt),$$
for all $p_1^{inc}$, $p_1^{e}$ s.t. $p_1^{inc}-p_1^{e}+(2-y)t \neq 0$.

$p=t+c$ is a standard Hotelling result. The new term of the price can be both negative and positive, which is not easy to pinpoint. As we assumed the entrant was chosen in the first period, we know that $p_1^{inc}-p_1^{e}-yt>-2tx$. Then it can take negative values from the interval $(-2tx,0)$ and any positive values. Using the expressions for the price from higher above, we derive equilibrium profits for both incumbent and entrant, should this outcome occur:
$$\pi_2^{inc} = \frac{(t+2/3(p_1^{inc}-p_1^{e}-yt))^2}{p_1^{inc}-p_1^{e}+(2-y)t} $$
$$\pi_2^{e} = \frac{(t+1/3(p_1^{inc}-p_1^{e}-yt))^2}{p_1^{inc}-p_1^{e}+(2-y)t} $$
Note that both cases are possible: that profit of incumbent is higher than profit of entrant, and vice versus.


If, however, \pmb{$f\leq e$}, the situation is quite different. 

\begin{tikzpicture}
    % entrant 
    
    \draw[step=0.4cm,gray,very thin] (-2,-1) grid (3.6,2) node[anchor=north east, black] {entrant};
    \draw[thick] (-1.6,0) -- (2.4,0);
    \draw[thick, ->] (-1.6,0) -- (-1.6,1.6);
    \draw (6.75,0.8) -- (6.85,0.8) node[anchor=east] {1};
    \draw (-1.55,0.8) -- (-1.65,0.8) node[anchor=east] {1};
    \draw[thick, dashed] (2.4,0) -- (3.2,0);
    \draw (-1.6,1pt) -- (-1.6,-1pt) node[anchor=north] {0};
    \draw (2.4,1pt) -- (2.4,-1pt) node[anchor=north] {1};
    \draw (3.2,1pt) -- (3.2,-1pt) node[anchor=north] {y};
    \draw (1.2,1pt) -- (1.2,-1pt) node[anchor=north] {e};
    \draw (0,1pt) -- (0,-1pt) node[anchor=north] {f};
	\draw[pattern=north east lines, pattern color=blue] (0,0) rectangle (2.4,0.8);
    \draw[pattern=north west lines, pattern color=red] (1.2,0) rectangle (2.4,0.8);
    
    
    
    % incumbent
    
    \draw[step=0.4cm, gray, very thin] (6.4,-1) grid (12,2) node[anchor=north east, black] {incumbent};
    \draw[thick] (6.8,0) -- (10.8,0);
    \draw[thick, ->] (6.8,0) -- (6.8,1.6);
     \draw[thick, dashed] (10.8,0) -- (11.6,0);
    \draw (6.8,1pt) -- (6.8,-1pt) node[anchor=north] {0};
    \draw (6.75,0.8) -- (6.85,0.8) node[anchor=east] {1};
    \draw (10.8,1pt) -- (10.8,-1pt) node[anchor=north] {1};
    \draw (11.6,1pt) -- (11.6,-1pt) node[anchor=north] {y};
    \draw (8.4,1pt) -- (8.4,-1pt) node[anchor=north] {f};
    \draw (9.6,1pt) -- (9.6,-1pt) node[anchor=north] {e};
    \draw[pattern=north east lines, pattern color=blue] (6.8,0) rectangle (8.4,0.8);
    \draw[pattern=north west lines, pattern color=red] (9.6,0) rectangle (10.8,0.8);

\end{tikzpicture}
    
As both producers know for sure that $x>f$, in the second case entrant sells his good with probability 1, whereas the incumbent does not sell.
Then the entrant puts the price as high as he can, without running off the consumer, i.e. keeping $f\leq e$. That denotes a whole set of possible solutions, where prices of both incumbent and entrant are compliant with this condition and the consumer's participation constraint $v-p^{inc}_2-t(1-x)>0$. As far as I understand, the equilibrium prices are not among them, as f depends on both the incumbent's and entrant's price. The entrant wants to put the price as high as possible, however, the incumbent answers that action with decreasing his price until $f>e$ again as for him it is the only way to gain profit. Is it possible for the entrant to set such a price so as to block the incumbent from entering the market? The lowest possible price he can set is $p^{e}_2 = c$. Suppose he sets the price c, then the incumbent's price needed to keep the equilibrium in $f\leq e$ should be bigger or equal $c+p^{inc}_1-p^{e}_1 + t(1-y)$. Then suppose the incumbent also sets the price as low as possible. Then the condition for the entrant taking all of the market (i.e. attracting the consumer) is $p^{inc}_1-p^{e}_1 - t(y-1) \leq 0$. If y=1, then this condition is simply $p^{inc}_1 \leq p^{e}_1$; and any $y>1$ actually make the condition hold with less of a gap between two prices. This makes sense as this condition is simply that e, the expected indifferent consumer location from the first period, is closer to 1, which implies, since the entrant was chosen nonetheless, that x itself is much closer to entrant than incumbent. In this case the incumbent can't actually attract the consumer as it would require setting their price lower than c. The entrant, on the other hand, can actually increase their price by up to $|p^{inc}_1 - p^{e}_1 - t(y-1)|$ as long as the value under the modulus sign is negative. Notice also that from the intuition above solution inside the $f\leq e$ case is only possible if $p^{inc}_1 - p^{e}_1 - t(y-1) \leq 0$ holds.

Then the equilibrium is:
$$p^{e}_2  =  c - p^{inc}_1 + p^{e}_1 + t(y-1) 
$$
$$p^{inc}_2 \in (c, \infty)$$

and the profit of incumbent is always 0, where
as $\pi^{e}_2 = p^{e}_1 - p^{inc}_1 + t(y-1)$. Interestingly enough, the profit of entrant increases with y.
 


If \textbf{entrant is not chosen} in the first period, the situation is almost symmetric. Now both producers know that $p_1^{inc}+t*x \leq p_1^{e}+t(y-x)$, meaning that $x \leq \frac{p_1^{e}-p_1^{inc}+yt}{2t}$. Let us denote $\frac{p_2^{e}-p_2^{inc}}{2t}+y/2$ as d and keep the notation for e from previous section. As the entrant was not seen, the consumer chooses incumbent in the second period if $Pr(x \leq \frac{p_2^{e}-p_2^{inc}}{2t}+y/2 | p_1^{inc}+t*x \leq p_1^{e}+t(y-x)) = Pr(x \leq d | x \leq e)$ and the entrant if $Pr(x > \frac{p_2^{e}-p_2^{inc}}{2t}+y/2 | p_1^{inc}+t*x \leq p_1^{e}+t(y-x)) = Pr(x > d| x \leq e)$.  Then we have yet again two cases: when $d \geq e$ and $e<d$. We denote again with red colour the condition on e and with blue colour the condition on d.

Let's look at some illustrations for $d \geq e$.

\begin{tikzpicture}
   % entrant 
    
    \draw[step=0.4cm,gray,very thin] (-2,-1) grid (3.6,2) node[anchor=north east, black] {entrant};
    \draw[thick] (-1.6,0) -- (2.4,0);
    \draw[thick, ->] (-1.6,0) -- (-1.6,1.6);
    \draw (-1.55,0.8) -- (-1.65,0.8) node[anchor=east] {1};
    
    \draw[thick, dashed] (2.4,0) -- (3.2,0);
    \draw (-1.6,1pt) -- (-1.6,-1pt) node[anchor=north] {0};
    \draw (2.4,1pt) -- (2.4,-1pt) node[anchor=north] {1};
    \draw (3.2,1pt) -- (3.2,-1pt) node[anchor=north] {y};
    \draw (1.2,1pt) -- (1.2,-1pt) node[anchor=north] {d};
    \draw[label distance=10mm] (0,1pt) -- (0,-1pt) node[anchor=north] {e};
    \draw[pattern=north east lines, pattern color=red] (-1.6,0) rectangle (0,0.8);
    \draw[pattern=north west lines, pattern color=blue] (1.2,0) rectangle (2.4,0.8);

    % incumbent
    
    \draw[step=0.4cm,gray,very thin] (6.4,-1) grid (12,2) node[anchor=north east, black] {incumbent};
    \draw[thick] (6.8,0) -- (10.8,0);
    \draw[thick, ->] (6.8,0) -- (6.8,1.6);
    \draw (6.75,0.8) -- (6.85,0.8) node[anchor=east] {1};
    \draw[thick, dashed] (10.8,0) -- (11.6,0);
    \draw (6.8,1pt) -- (6.8,-1pt) node[anchor=north] {0};
    \draw (10.8,1pt) -- (10.8,-1pt) node[anchor=north] {1};
    \draw (11.6,1pt) -- (11.6,-1pt) node[anchor=north] {y};
    \draw (8.4,1pt) -- (8.4,-1pt) node[anchor=north] {e};
    \draw (9.6,1pt) -- (9.6,-1pt) node[anchor=north] {d};
    \draw[pattern=north east lines, pattern color=red] (6.8,0) rectangle (8.4,0.8);
    \draw[pattern=north west lines, pattern color=blue] (6.8,0) rectangle (9.6,0.8);
    

\end{tikzpicture}

Notice that for $d \geq e$ entrant has no chance of attracting the consumer, whereas the incumbent always does. Let's try our intuition from the previous section. The optimal price for the incumbent would have been the one, under which d=e, as the $d \geq e$ is obviously beneficial to the entrant and the higher the price, the bigger are the profits. However, incumbent can always move d away from this point into $d < e$ territory, where he has a non-zero probability to attract the consumer by lowering his own price. Can incumbent set a price s.t. entrant can no longer hinder him? Suppose entrant sets the price c, and since in this case the transportation cost part of the price actually cancels itself, we have: $ p_2^{inc}-c \leq p_1^{inc}-p_1^{e}$. So the maximal markup that the entrant can afford while keeping the consumer to himself is equal to $p_1^{inc}-p_1^{e}$, as long as $p_1^{inc} \geq p_1^{e}$. Similar to the previous case the optimal strategies are then:

Then the equilibrium is:
$$p^{inc}_2  =  c + p^{inc}_1 - p^{e}_1 
$$
$$p^{e}_2 \in (c, \infty)$$

and the profit of incumbent is always 0, whereas $\pi^{inc}_2 = p^{inc}_1  - p^{e}_1$. 

If $d < e$, however, the field is a bit more even.

\begin{tikzpicture}
   % entrant 
    
    \draw[step=0.4cm,gray,very thin] (-2,-1) grid (3.6,2) node[anchor=north east, black] {entrant};
    \draw[thick] (-1.6,0) -- (2.4,0);
    \draw[thick, dashed] (2.4,0) -- (3.2,0);
    \draw[thick, ->] (-1.6,0) -- (-1.6,1.6);
    \draw (-1.55,0.8) -- (-1.65,0.8) node[anchor=east] {1};
    \draw (-1.6,1pt) -- (-1.6,-1pt) node[anchor=north] {0};
    \draw (2.4,1pt) -- (2.4,-1pt) node[anchor=north] {1};
    \draw (3.2,1pt) -- (3.2,-1pt) node[anchor=north] {y};
    \draw (1.2,1pt) -- (1.2,-1pt) node[anchor=north] {e};
    \draw[label distance=10mm] (0,1pt) -- (0,-1pt) node[anchor=north] {d};
    \draw[pattern=north east lines, pattern color=red] (-1.6,0) rectangle (1.2,0.8);
    \draw[pattern=north west lines, pattern color=blue] (0,0) rectangle (2.4,0.8);

    % incumbent
    
    \draw[step=0.4cm,gray,very thin] (6.4,-1) grid (12,2) node[anchor=north east, black] {incumbent};
    \draw[thick] (6.8,0) -- (10.8,0);
    \draw[thick, ->] (6.8,0) -- (6.8,1.6);
    \draw (6.75,0.8) -- (6.85,0.8) node[anchor=east] {1};
    \draw[thick, dashed] (10.8,0) -- (11.6,0);
    \draw (6.8,1pt) -- (6.8,-1pt) node[anchor=north] {0};
    \draw (10.8,1pt) -- (10.8,-1pt) node[anchor=north] {1};
    \draw (11.6,1pt) -- (11.6,-1pt) node[anchor=north] {y};
    \draw (8.4,1pt) -- (8.4,-1pt) node[anchor=north] {d};
    \draw (9.6,1pt) -- (9.6,-1pt) node[anchor=north] {e};
    \draw[pattern=north east lines, pattern color=blue] (6.8,0) rectangle (8.4,0.8);
    \draw[pattern=north west lines, pattern color=red] (6.8,0) rectangle (9.6,0.8);
    

\end{tikzpicture}

From the graph we see that the demand the entrant faces is equal to $Pr(x>d|x \leq e)=\frac{e-d}{e}$. Demand of the incumbent, on the other hand, is is equal to $Pr(x \leq d| x \leq e) = \frac{d}{e}$. Substituting these equations into the corresponding profit functions and solving for optimal price, we derive the following equilibrium prices:
$$ p_2^{inc} = c+ 1/3(p_1^{e}-p_1^{inc}) + 2/3yt$$
$$ p_2^{e} = c + 2/3(p_1^{e}-p_1^{inc})+ 1/3yt,$$
for all $p_1^{inc}$, $p_1^{e}$ s.t. $p_1^{e}-p_1^{inc}+yt \neq 0$.
Notice that if y=1 the prices look very similar to the entrant-is-noticed, $f > e$ case (also with y=1).

The profits in this case are:

$$\pi_2^{inc} = \frac{(2/3yt+1/3(p_1^{e}-p_1^{inc}))^2}{p_1^{e}-p_1^{inc}+yt} $$
$$\pi_2^{e} = \frac{(1/3yt+2/3(p_1^{e}-p_1^{inc}))^2}{p_1^{e}-p_1^{inc}+yt} $$

\textbf{This part I am entriely unsure about.}

We know that the "corner" solutions $f \leq e$ and $d>e$ are only possible under certain conditions. Then we can assume that there might be two types of possible equilibria for the whole game: under $p^{inc}_1 - p^{e}_1 - t(y-1) \leq 0$ a corner solution with entrant taking the market and under $p_1^{inc} \geq p_1^{e}$ the corner solution with incumbent taking all the market. Then, under the opposite conditions we have interior solutions, with either entrant or incumbent having an advantage for having been chosen in the first period. 

As in the previous subsection, we are inputting into the expected profit function the probability of entrant being chosen. We will optimize profit assuming each condition combination holds and then check if that is indeed the case.

\begin{enumerate}
	\item $p^{inc}_1 \in [p^{e}_1, p^{e}_1 + t(y-1)]$
	
	Since both the entrant and the incumbent expect that setting the prices to follow this condition they will always end up in the corner solution, there is no need for conditional probability.
	Then the expected profits are: 
	
	$$E(\pi_0^{e}) = Pr(x>e)(p^{e}_1 -c) + \beta( Pr(x>e)(-p^{inc}_1 + p^{e}_1 + t(y-1)) + (Pr(x \leq e)*0)  $$
	
	$$E(\pi_0^{inc}) = Pr(x \leq e)(p^{inc}_1 -c) + \beta( Pr(x \leq e)(-p^{inc}_1 + p^{e}_1) + Pr(x>e)*0) $$
	
	
	\item 	$p^{inc}_1 \in (p^{e}_1 + t(y-1), \infty)$
	
	$$E(\pi_0^{e}) = Pr(x>e)(p^{e}_1 -c) + \beta( Pr(x>e)(\frac{(t+1/3(p_1^{inc}-p_1^{e}-yt))^2}{p_1^{inc}-p_1^{e}+(2-y)t}) + (Pr(x \leq e)*0)  $$
	
	$$E(\pi_0^{inc}) = Pr(x \leq e)(p^{inc}_1 -c) + \beta( Pr(x \leq e)(\frac{(t+2/3(p_1^{inc}-p_1^{e}-yt))^2}{p_1^{inc}-p_1^{e}+(2-y)t}) + Pr(x>e)*0) $$
	
	\item $p^{inc}_1 \in (-\infty, p^{e}_1)$
	$$E(\pi_0^{e}) = Pr(x>e)(p^{e}_1 -c) + \beta( Pr(x>e)(\frac{(t+1/3(p_1^{inc}-p_1^{e}-yt))^2}{p_1^{inc}-p_1^{e}+(2-y)t}) + (Pr(x \leq e)*\frac{(2/3yt+1/3(p_1^{e}-p_1^{inc}))^2}{p_1^{e}-p_1^{inc}+yt})  $$
	
	$$E(\pi_0^{inc}) = Pr(x \leq e)(p^{inc}_1 -c) + \beta( Pr(x \leq e)(\frac{(t+2/3(p_1^{inc}-p_1^{e}-yt))^2}{p_1^{inc}-p_1^{e}+(2-y)t}) + Pr(x>e)*\frac{(1/3yt+2/3(p_1^{e}-p_1^{inc}))^2}{p_1^{e}-p_1^{inc}+yt}) $$
	
	
	
\end{enumerate}

Optimizing for each condition we get up to 3 possible equilibria, which we then check for corresponding to each condition.

\textbf{Is this a valid approach or am I doing something very bad? If the latter, can you recommend me how to proceed or what I can read about this?}

\end{document}
