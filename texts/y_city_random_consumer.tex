\documentclass{article}
\usepackage[utf8]{inputenc}
\usepackage[allcolors=blue]{hyperref}
\usepackage[left=2.3cm, right=2.3cm, scale=0.75,top=2cm, bottom=2cm]{geometry}
\usepackage[all]{hypcap}
\usepackage{caption}
\usepackage{natbib} % Use natbib for citing
\usepackage{amsfonts}
\usepackage{amsmath}
\usepackage{graphicx} 
\usepackage{tikz}
\graphicspath{{../graphs/}}
\usetikzlibrary{patterns}




\title{Random consumer in a perceived [0,y] linear city under duopoly}
\author{Iana Gerina}
\date{\today}
%\geometry{
% a4paper,
% left=20mm,
% top=15mm,
% right=20mm,
% bottom=15mm,
% }

\begin{document}

\maketitle


\section{Introduction}
Two producers on the market: one is incumbent and one is entrant. Suppose location of incumbent and entrant is set: 0 for incumbent and 1 for entrant; however, the consumer perceives the location of entrant as $y>1$ in the first period. The strategies of the producers are then prices: $p_{inc}$ and $p_{e}$ - price of incumbent and entrant respectively. Consumer chooses between the option offered by the incumbent and the entrant (and an outside option, whose value we set to 0), and if in the first period the entrant is chosen, his true location is revealed to the consumer in the second period.

As the original model talks about a set of consumers, we need to adjust the demand estimation accordingly. We consider only x high enough that the consumer must buy one of the products (which is often the case for bus tickets). The consumer's utility is:

    $$ U = \begin{cases}
    v - t*x - p_{inc} &  \text{if } v-t*x - p_{inc} \geq v - t(y-x) - p_{e}\\
    v - t*x - p_{e} &  \text{if } v-t*x - p_{inc} < v - t(y-x) - p_{e}
    \end{cases}
$$

Then, demand for incumbent's good is:

$$d_{inc} = \begin{cases}
    1 &  \text{if } x \leq \frac{p_{e}-p_{inc}}{2t} + 1/2y\\
    0 &  \text{if } x > \frac{p_{e}-p_{inc}}{2t} + 1/2y
    \end{cases}
$$
 and vice versus for demand of the entrant.
 
In a one period game incumbent then solves:
$$\max_{p^{inc}} (p^{inc}-c)* Pr(x \leq \frac{p_{e}-p_{inc}}{2t} + 1/2y),$$
which can be simplified to:
$$\max_{p^{inc}} (p^{inc}-c)(\frac{p_{e}-p_{inc}}{2t} + 1/2y)$$
and entrant's profit-maximising problem to:
$$\max_{p^{e}} (p^{e}-c)(\frac{p_{inc}-p_{e}}{2t} + 1 - 1/2y)$$

\section{Baseline two-period model}
    
    Consider a following market interaction: in each period one consumer enters the market to buy one unit of good. In each period the consumer  is located differently with his location x being a random draw from U[0,1]. Once consumer is located s.t. the entrant's good is his best option and thus buys entrant's good, in the next periods he perceives the true location of the entrant.
    
    We have a two-period game. Then from the first period point of view, the probability of the entrant being seen in the second period depends on the probability of him being the best option in the first period. Thus, we can write the optimization problem in the first period for both incumbent and entrant as follows:
    
    $$ \begin{aligned}
    \max_{p^{inc}_1, p^{inc}_2} {}
    & \quad (p^{inc}_1-c)* Pr(x_1 \leq \frac{p^{e}_1-p^{inc}_1}{2t} + 1/2y) + \\
    & \quad + \beta((p^{inc}_2-c)(Pr(x_1 \leq \frac{p^{e}_1-p^{inc}_1}{2t} + 1/2y)(Pr(x_2 \leq \frac{p^{e}_2-p^{inc}_2}{2t} + 1/2y)) + \\
    & \quad + (Pr(x_1 > \frac{p^{e}_1-p^{inc}_1}{2t} + 1/2y))(Pr(x_2 \leq \frac{p^{e}_2-p^{inc}_2}{2t} + 1/2)))
    \end{aligned}
    $$

    $$ \begin{aligned}
    \max_{p^{e}_1, p^{e}_2} {}
    & \quad (p^{e}_1-c)* Pr(x_1 > \frac{p^{e}_1-p^{inc}_1}{2t} + 1/2y) + \\
    & \quad + \beta((p^{e}_2-c)(Pr(x_1 \leq \frac{p^{e}_1-p^{inc}_1}{2t} + 1/2y)(Pr(x_2 \leq \frac{p^{inc}_2-p^{e}_2}{2t} + 1 - 1/2y)) + \\
    & \quad + (Pr(x_1 > \frac{p^{e}_1-p^{inc}_1}{2t} + 1/2y))(Pr(x_2 \leq \frac{p^{inc}_2-p^{e}_2}{2t} + 1/2))),
    \end{aligned}
    $$
where $x_1$, $x_2$ are locations of the consumer in each period. 

\subsection{Derivation}
    
Solving by backward induction, we first discuss the equilibrium conditions in the second period. 
If entrant is noticed, the solution corresponds to classic Hotelling and the profit is then standard $\pi_{2}^{inc} = \pi_{2}^{e}= t/2$. However, if entrant is not noticed, equilibrium prices are as follows:
$$
p^{inc}_2 =1/3t(y+ 2) + c $$
$$ p^{e}_2 =4/3t - 1/3yt + c$$
To make sure that the profits don't go into the negative values, $ p^{e}_2$ should be less than 4, otherwise entrant does not enter the market. Then the profits are $\pi_{2}^{e} = 1/18*t(4-y)^2$ and $\pi_{2}^{inc}=1/18*t(2+y)^2$, respectively. The profit of incumbent is increasing in y and the profit of entrant is decreasing in y for all the allowed values of y ($1<y\leq4$), exponentially. Also notice that if y=1, the profits equations collapse to the classic Hotelling profits. 

Now each producer optimizes their behaviour in the first period, assuming that in each state (multiplied by corresponding probability of reaching it) in the second period they will behave optimally. 

	


 $$
    \max_{p^{inc}_1} {}
   (p^{inc}_1-c)(\frac{p^{e}_1-p^{inc}_1}{2t} + 1/2y) 
    + \beta((\frac{p^{e}_1-p^{inc}_1}{2t} + 1/2y)(\frac{1}{18}t(2+y)^2) + 
    1/2t(\frac{p^{inc}_1-p^{e}_1}{2t} + 1 - 1/2y))
    $$
    
    $$
   \max_{p^{e}_1} {}
   (p^{e}_1-c)(\frac{p^{inc}_1-p^{e}_1}{2t} + 1 -1/2y) 
    + \beta((\frac{p^{e}_1-p^{inc}_1}{2t} + 1/2y)(\frac{1}{18}t(4-y)^2) + 
    1/2t(\frac{p^{inc}_1-p^{e}_1}{2t} + 1 - 1/2y))
    $$

The equilibrium prices in the first period are then:
$$
p^{inc}_1 =1/3t(y+ 2) + c - \frac{\beta t }{54}(y-1)(17+y) $$
$$ p^{e}_1 =4/3t - 1/3yt + c - \frac{\beta t}{54}(y-1)(19-y)$$
Notice that for both the prices the RHS until the $ \beta$ component correspond to our second period equilibrium, the 'unnoticed entrant' state, which in turn collapses to the standard result, when y=1. There is also no dynamic effect in the price, if y=1, as the component becomes 0 - also quite logical, as the dynamic effect appears because of the difference between two states' profits in the second period. 

\subsection{Comparative statics}

From the graph it is obvious that high learning cost, modelled via y, has an obvious negative impact on entrant's profit and price. Interestingly, higher valuation of the future profits results in the increase of them for both producers, which makes sense as the price is chosen in a dynamic way so as to receive optimal profits in the secon period as well. The myopic producer does not perceive that there are profits to be had in the second period at all. The most curious is the optimal price of the producer in the non-myopic case: it stays almost the same and even decreases with higher values of y if producer is maximally forward-looking. The incumbent does not dare increase the price in the first period as he is afraid of entrant being noticed; moreso when his loss from entrant being tested is the highest with y the highest. The entrant is also forced to lower his price more and more as compared to the myopic case, as he has a lot to gain from being tested, going for high values of y to 0. The profits of incumbent are increasing exponentially with y regardless, as he is reaping benefits from high probability of him being the closest in consumer's eyes. 

\begin{center}
    \includegraphics[scale=0.7]{random_consumer_location.png}
\end{center}


\includegraphics[scale=0.7]{random_consumer_location_prices.png}
\includegraphics[scale=0.7]{random_consumer_location_profits.png}

\includegraphics[scale=0.7]{baseline_profits_incumbent.png}
\includegraphics[scale=0.7]{baseline_profits_entrant.png}
\includegraphics[scale=0.7]{baseline_prices_incumbent.png}\\
\includegraphics[scale=0.7]{baseline_prices_entrant.png}

\section{Probability $\alpha$ of being perceived at the right location}

We build on assumptions of the base model in this section, but consider additionally that there is a chance that the entrant's true location will be noticed in each period. The probability of this happenning we define as $\alpha \in [0,1]$. 

The intuition behind this assumption could be that entrant invests in advertising and thus makes his presence known to the consumer (which could be additiionally modelled by assuming different costs for incumbent and entrant); or, the entrant is a big company, whose products in other markets are known to consumers and they are thus less reluctant to try out the new good. In the case of bus lines, "Blablabus" is a daughter company of "Blablacar", which has been used by many for travelling and thus has already some consumers familiar with it. These consumers would not be as deterred by the company's newcomer's status as other consumers probably are.

\subsection{Derivation}

Consider a one-period model. If $\alpha = 0$, then the model collapses to the one in the introduction. If $\alpha = 1$, on the other hand, then the model is simply the original Hotelling model and does not require additional derivation. 

For ease of calculation, set c=0. If $alpha \in (0,1)$, the producers solve the following maximisation problems:

$$\max_{p^{inc}} p^{inc}* (\alpha Pr(x \leq \frac{p_{e}-p_{inc}}{2t} + 1/2) + (1-\alpha)Pr(x \leq \frac{p_{e}-p_{inc}}{2t} + 1/2y))$$

$$\max_{p^{e}} p^{e}* (\alpha Pr(x > \frac{p_{e}-p_{inc}}{2t} + 1/2) + (1-\alpha)Pr(x > \frac{p_{e}-p_{inc}}{2t} + 1 - 1/2y))$$

Solving for equilibrium price we get:
$$p^{inc} = \alpha t + \tfrac{(1-\alpha)(2+y)t}{3}$$
$$p^{e} = \alpha t + \tfrac{(1-\alpha)(4-y)t}{3}$$

You can see that if $\alpha = 0$ the solution corresponds to that of the base model second period maximisation, and if $\alpha = 1$ the prices collapse to Hotelling linear city $p=t$.

The corresponding profits are $\pi^{inc} = t/2(\alpha + \tfrac{(1-\alpha)(2+y)}{3})^2$ and $\pi^{e} = t/2(\alpha + \tfrac{(1-\alpha)(4-y)}{3})^2$. Notice that the profits for $\alpha=0$ case correspond to the base model profits, and for $\alpha=0$ - to the standard Hotelling.

Now we expand the model to a two-period one.

We assume the following information structure: while the probability of the consumer seeing the entrant is known to both producers and the same in each period, they do not know if the consumer is noticed. However, the choice of the consumer is public, so if entrant was chosen in the first period, both producers know that the true location of the entrant has been revealed. The intuition behind this setup is that while some ticket buyers may have heard about the entrant already, the bus lines have no way of knowing that; but they can perceive, if the consumer has bought tickets before; although, in truth, most bus lines do not engage in this level of price discrimination.

The discounted sum of profits of the consumer is then:

 $$ \begin{aligned}
    \max_{p^{inc}_1, p^{inc}_2} {}
    & \quad p^{inc}_1(\alpha Pr(x_1 \leq \frac{p^{e}_1-p^{inc}_1}{2t} + 1/2) + (1-\alpha)Pr(x_1 \leq \frac{p^{e}_1-p^{inc}_1}{2t} + 1/2y)) \\
    & \quad + \beta(p^{inc}_2((\alpha Pr(x_1 \leq \frac{p^{e}_1-p^{inc}_1}{2t} + 1/2) + (1-\alpha)Pr(x_1 \leq \frac{p^{e}_1-p^{inc}_1}{2t} + 1/2y))* \\
    & \quad (\alpha Pr(x_2 \leq \frac{p^{e}_2-p^{inc}_2}{2t} + 1/2) + (1-\alpha)Pr(x_2 \leq \frac{p^{e}_2-p^{inc}_2}{2t} + 1/2y)) +\\
    & \quad \alpha Pr(x_1 > \frac{p^{e}_1-p^{inc}_1}{2t} + 1/2) + (1-\alpha)Pr(x_1 > \frac{p^{e}_1-p^{inc}_1}{2t} + 1/2y))*\\
    & \quad Pr(x_2 > \frac{p^{e}_2-p^{inc}_2}{2t} + 1/2))
    \end{aligned}
    $$

where $x_1$, $x_2$ are locations of the consumer in each period. Sum of discounted profits for entrant is symmetric. The expression can be simplified, however, we first solve for optimum in second period to put in values into the formula.

If the entrant is chosen in the first period, the interaction corresponds to normal Hotelling linear city, which gives $p^{inc}_2=p^{e}_2 = t$ and the profits $\pi^{inc}_2 = \pi^{e}_2 = 1/2t$. If the entrant is not chosen, both producers need to consider again that consumer may or may not know the true location of the entrant. Then the interaction corresponds to the one period interaction we described for one-period game with $p^{inc}_2 = \alpha t + \tfrac{(1-\alpha)(2+y)t}{3}$ and $p^{e}_2 = \alpha t + \tfrac{(1-y)(4-y)t}{3}$. The profits are also equal to those in the one period game: $\pi^{inc} = t/2(\alpha + \tfrac{(1-\alpha)(2+y)}{3})^2$ and $\pi^{e} = t/2(\alpha + \tfrac{(1-\alpha)(4-y)}{3})^2$.

Substitute those profits into the discounted sum of profits maximisation problem:

$$ \begin{aligned}
    \max_{p^{inc}_1} {}
    & \quad p^{inc}_1(\alpha Pr(x_1 \leq \frac{p^{e}_1-p^{inc}_1}{2t} + 1/2) + (1-\alpha)Pr(x_1 \leq \frac{p^{e}_1-p^{inc}_1}{2t} + 1/2y)) + \\
    & \quad + \beta((\alpha Pr(x_1 \leq \frac{p^{e}_1-p^{inc}_1}{2t} + 1/2) + (1-\alpha)Pr(x_1 \leq \frac{p^{e}_1-p^{inc}_1}{2t} + 1/2y))* \\
    & \quad *(\alpha + \tfrac{(1-\alpha)(2+y)}{3})^2 t/2+ \tfrac{\alpha t}{2} Pr(x_1 > \frac{p^{e}_1-p^{inc}_1}{2t} + 1/2) + \\
    & \quad + \tfrac{(1-\alpha)t}{2}Pr(x_1 > \frac{p^{e}_1-p^{inc}_1}{2t} + 1/2y))
    \end{aligned}
    $$
   
The profit of entrant is symmetric:


$$ \begin{aligned}
    \max_{p^{e}_1} {}
    & \quad p^{e}_1(\alpha Pr(x_1 > \frac{p^{e}_1-p^{inc}_1}{2t} + 1/2) + (1-\alpha)Pr(x_1 > \frac{p^{e}_1-p^{inc}_1}{2t} + 1/2y)) + \\
    & \quad + \beta((\alpha Pr(x_1 \leq \frac{p^{e}_1-p^{inc}_1}{2t} + 1/2) + (1-\alpha)Pr(x_1 \leq \frac{p^{e}_1-p^{inc}_1}{2t} + 1/2y))* \\
    & \quad *(\alpha + \tfrac{(1-\alpha)(4-y)}{3})^2 t/2+ \tfrac{\alpha t}{2} Pr(x_1 > \frac{p^{e}_1-p^{inc}_1}{2t} + 1/2) + \\
    & \quad + \tfrac{(1-\alpha)t}{2}Pr(x_1 > \frac{p^{e}_1-p^{inc}_1}{2t} + 1/2y))
    \end{aligned}
    $$


Both can be simplified substantially. Solving for optimal $p^{inc}_1$ and $p^{e}_1$ we get:

$$p^{inc}_1 = \alpha t + \tfrac{(1-\alpha)(2+y)t}{3} + \tfrac{\beta t}{54}(1-\alpha)(1-y)(17+ \alpha +y(1-\alpha))$$
$$p^{e}_1 = \alpha t + \tfrac{(1-\alpha)(4-y)t}{3} + \tfrac{\beta t}{54}(1-\alpha)(1-y)(19 - \alpha -y(1-\alpha))$$

Notice that the first, non-dynamic part of both expressions looks exactly like the corresponding prices for the second period. If $\alpha=1$ the prices collapse to the corresponding prices in the baseline model. If $\alpha=0$, then the prices are just standard Hotelling prices.

\subsection{Comparative statics}

(COMPARATIVE STATICS)

\section{Network effects modelled with $\gamma$}

While the previous setup tried to reflect possible advertising effects or brand loyalty, now we consider possible network effects. While in the first period the entrant starts out completely unknown to all consumers in the baseline model, it might be prudent to assume that consumers talk between themselves. If the first-period consumer choose the entrant, it might be easier for the second period consumer to try out the new company's goods, as he knows about the company from second-hand experience.

Consider the baseline model with a few additional assumptions. In the first period the interaction is that of the baseline model, however, if the entrant was chosen, in the second period there is a probability $\\gamma$ that the entrant's true location is known to the consumer. The information structure is such: while the choice of the consumer is public, neither producer in the second period know if the consumer knows the entrant's location. Once the entrant is chosen by one consumer, next consumer knows his true location with probability $\gamma$.

\subsection{Derivation}

Mathematically, the setup is very close to both the previous section and the baseline model, which allows us to simplify the inference considerably. One-period game in this setup corresponds to that of baseline model, so we start directly with the two-period game.

If the entrant was chosen in the first period, then the maximisation problem is one described in the previous section for unrevealed true location of entrant state. We know that the optimal prices are in this case: $p^{inc}_2 = \gamma t + \tfrac{(1-\gamma)(2+y)t}{3}$ and $p^{e}_2 = \gamma t + \tfrac{(1-y)(4-y)t}{3}$ with the corresponding profits $\pi^{inc} = t/2(\gamma + \tfrac{(1-\gamma)(2+y)}{3})^2$ and $\pi^{e} = t/2(\gamma + \tfrac{(1-\gamma)(4-y)}{3})^2$. If the entrant was not chosen, then the interaction in the first period repeats as per baseline model, which means that: $p^{inc}_2 =1/3t(y+ 2)$ and $ p^{e}_2 =4/3t - 1/3yt$ and the profits are:  $\pi_{2}^{e} = 1/18*t(4-y)^2$ and $\pi_{2}^{inc}=1/18*t(2+y)^2$. 

The producers then maximize:

$$ \begin{aligned}
    \max_{p^{inc}_1} {}
    & \quad p^{inc}_1(Pr(x_1 \leq \frac{p^{e}_1-p^{inc}_1}{2t} + 1/2y)) + \beta(\tfrac{t}{18} Pr(x_1 \leq \frac{p^{e}_1-p^{inc}_1}{2t} + 1/2y)(2+y)^2 + 	\\
    & \quad \tfrac{t}{2}Pr(x_1 > \frac{p^{e}_1-p^{inc}_1}{2t} + 1/2y)(\gamma + \tfrac{(1-\gamma)(2+y)}{3})^2)
    \end{aligned}
    $$
   
$$ \begin{aligned}
    \max_{p^{e}_1} {}
    & \quad p^{e}_1(Pr(x_1 > \frac{p^{e}_1-p^{inc}_1}{2t} + 1/2y)) + \beta(\tfrac{t}{18} Pr(x_1 \leq \frac{p^{e}_1-p^{inc}_1}{2t} + 1/2y)(4-y)^2 + 	\\
    & \quad \tfrac{t}{2}Pr(x_1 > \frac{p^{e}_1-p^{inc}_1}{2t} + 1/2y)(\gamma + \tfrac{(1-\gamma)(4-y)}{3})^2)
    \end{aligned}
    $$
Solving for optimal price in the first period we get: 

$$p^{inc}_1 = \tfrac{(2+y)t}{3} + \tfrac{\alpha \beta t}{54}(1-y)(9\gamma + (2-\gamma)(8+y))$$
$$p^{e}_1 = \tfrac{(4-y)t}{3} + \tfrac{\gamma \beta t}{54}(1-y)(9 \gamma + (2-\gamma)(10-y))$$

Notice that when $\gamma = 1$ the prices are the same as for baseline model and for $\gamma = 0$ they are constant between periods, as there is no dynamic effect as the consumers never learn the entrant's true location.

\subsection{Comparative statics}



%\bibliography{literature.bib} % Filename of bibliography
%\bibliographystyle{apalike}
    
\end{document}
