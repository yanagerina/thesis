\documentclass{article}
\usepackage[utf8]{inputenc}
\usepackage[allcolors=blue]{hyperref}
\usepackage[left=2.3cm, right=2.3cm, scale=0.75,top=2cm, bottom=2cm]{geometry}
\usepackage[all]{hypcap}
\usepackage{caption}
\usepackage{natbib} % Use natbib for citing
\usepackage{amsfonts}
\usepackage{amsmath}
\usepackage{graphicx} 
\usepackage{tikz}
\graphicspath{{../graphs/}}
\usetikzlibrary{patterns}




\title{Each period: 1 consumer, two producers}
\author{Iana Gerina}
\date{\today}
%\geometry{
% a4paper,
% left=20mm,
% top=15mm,
% right=20mm,
% bottom=15mm,
% }

\begin{document}

\maketitle


\section{$\alpha$ - probability of seeing the entrant}

As a basis we take the standard linear city Hotelling model, where consumers are distributed uniformly on (0,1). The supply side is reperesented by two producers: the entrant and the incumbent. While the incumbent is known to the consumers, the entrant first needs to be noticed by them. 

Assume in each period only 1 consumer is looking to buy the good. Let us denote $\alpha \in (0,1)$ as the probability of entrant being noticed. Then, there are two states of the world in each period:

\begin{itemize}
    \item with probability $\alpha$ we arrive at state 1: duopoly, the demand that the producers face is reminiscent of standard Hotelling model;
    \item with probability $1-\alpha$ we arrive at state 2: monopoly, where consumer obly sees the incumbent.
\end{itemize}

Consider a one-period game. If $\alpha = 0$ the incumbent is always a monopolist. What is the pricing strategy under monopoly? The incumbent solves:
$$\max_{p^{inc}} (p^{inc}-c)* Pr(p^{inc}\leq v-t*x)$$

The demand the monopoly faces is then $\frac{v-p^{inc}}{t}$ as $Pr(p^{inc}\leq v-t*x) = Pr(x\leq \frac{v-p^{inc}}{t})$. Imposing market coverage condition $\frac{v-p^{inc}}{t} \geq  1$, we get $p \leq v-t$. The optimal price coming from maximisation problem is $p=v/2$ and for any $t \leq v/2$ the market coverage condition holds. However, it is obvious that under this condition $v-t \geq v/2$, so the incumbent strictly prefers to set the price $p^{inc} = v-t = \pi^{inc}$ as the probability of attracting the consumer is the same in both cases. Thus, to simplify analysis we restrict $t \leq v/2$. 

If, on the other hand, $\alpha=1$, the game is basically the standard linear city model with duopoly. The demands that incumbent and entrant face are, in same order:
$$ x =  \tfrac{p^{e}_2 - p^{inc}_2}{2t} + 1/2 $$
$$1-x = \tfrac{p^{inc}_2 - p^{e}_2}{2t} + 1/2$$

Yet again, we want to insure market coverage, which only occurs if $t<2v/3$. Since we already assumed $t \leq v/2$, no additional assumption is needed.

Let us expand the model to encompass two periods. The information structure is as follows: the probability of entrant being seen is known to both entrant and incumbent, and so is the event of entrant being noticed. Until that point both producers assume he was not and assign probability $\alpha$ to that event happening in each period. If consumer gets noticed in the first period, $\alpha =1$ in the second period.To simplify matters further, let us assume c=0 for both producers. We start from the second period.

As was mentioned above, we have two possible states: if entrant was noticed in the first period and if he wasn't. If \textbf{entrant was noticed}, the solution is straightforward, as it is the prices, demands and profits of standard Hotelling duopoly, that we already established. The equilibrium prices are then: $p^{inc}_2 = p^{e}_2 = t+c$ and as we establichd c=0, the price is then just equal to transportation cost. The profits in this case are 1/2t for each of the firms.

However, if the \textbf{entrant was not noticed}, the solution is less straightforward. With probability $\alpha$ the state is a duopoly with corresoonding duopolistic demands that the producers face and with probability $1- \alpha$ it's a monopoly with corresponding demand described above. 
The expected profits in the second period that they maximise are then:

$$ \pi^{inc}_2 = (1-\alpha)(v-t) +  \alpha(Pr(x \leq \tfrac{p^{e}_2 - p^{inc}_2}{2t}) + 1/2)$$

$$ \pi^{e}_2 = \alpha *p_2^{e}(Pr(x > \tfrac{p^{e}_2 - p^{inc}_2}{2t}) + 1/2)
$$

Solving for optimal price we get:

$$
p_2^{inc} = t + 4/3 \tfrac{1-\alpha}{\alpha}(v-t)t$$
$$
p_2^{e} = t + 2/3\tfrac{1-\alpha}{\alpha}(v-t)t
$$

Notice that for $\alpha = 1$ the prices of both incumbent and entrant are our duopoly prices from one-period game and for $\alpha -> 0$ the prices of both producers tend to infinity, with the incumbent's price increasing at a higher rate. 
Substituting optimal prices into the profit equation we get: 

$$ \pi^{inc}_2 = \tfrac{2-\alpha}{2t}(\tfrac{4v(1-\alpha) + 3\alpha t}{8-5\alpha})^2$$

$$ \pi^{e}_2 = \tfrac{\alpha}{2t}(\tfrac{(4-\alpha)t}{8-5\alpha} + \tfrac{2(1-\alpha)v}{8-5\alpha})^2 
$$
Notice that for $\alpha=0$ the profit of entrant is 0 and the profit of incumbent is the monopolist profit $\tfrac{v^2}{4t}$ and for $\alpha=1$ the profits are again the dupoly profits $1/2t$ for both producers.

Discounted sums of profits in the first period for both producers can be then written as:

 $$ \pi^{inc} = (1-\alpha)(\tfrac{v-p_1^{inc}}{t}*p_1^{inc}) +  \alpha p^{inc}_1(\tfrac{p^{e}_1 - p^{inc}_1}{2t} + 1/2) + \beta(Pr(x \leq \tfrac{p^{e}_1 - p^{inc}_1}{2t} + 1/2)( \tfrac{2-\alpha}{2t}(\tfrac{4v(1-\alpha) + 3\alpha t}{8-5\alpha})^2 + 1/2t(Pr(x > \tfrac{p^{inc}_1 - p^{e}_1}{2t} + 1/2)))  $$
   
 $$ \pi^{e} = \alpha p^{e}_1(Pr(x > \tfrac{p^{e}_1 - p^{inc}_1}{2t} + 1/2)) + \beta(Pr(x \leq \tfrac{p^{e}_1 - p^{inc}_1}{2t} + 1/2)*\tfrac{\alpha}{2t}(\tfrac{(4-\alpha)t}{8-5\alpha} + \tfrac{2(1-\alpha)v}{8-5\alpha})^2 + 1/2t(Pr(x > \tfrac{p^{e}_1 - p^{inc}_1}{2t} + 1/2))) $$

For ease of reading, let us denote $(\tfrac{(4v(1-\alpha) + 3\alpha t)}{8-5\alpha})^2$ as m and $(\tfrac{((4-\alpha)t + 2(1-\alpha)v)}{8-5\alpha})^2$ as n. Then, solving for optimal prices we get:

$$ p^{inc}_1 = \tfrac{3\alpha}{8-5\alpha}t + \tfrac{4(1-\alpha)}{8-5\alpha}v + \tfrac{\beta t}{2(8-5\alpha)} + \tfrac{\alpha \beta m - 2\beta(2-\alpha)n}{2t(8-5\alpha)}$$

$$ p^{e}_1 = \tfrac{t(4-\alpha)}{8-5\alpha} +\tfrac{2(1-\alpha)v}{8-5\alpha} + \tfrac{\alpha \beta m + \beta n (4-3\alpha)}{4t(8-5\alpha)} - \beta t \tfrac{4-3\alpha}{2\alpha(8-5\alpha)} 
$$


\textbf{(I would really appreciate if you could look into the price equations: I don't see a way to simplify them, and that is suspicious to me)}.

Notice that when $\alpha = 1$ the price of the incumbent is simply t like in Hotelling model and that both prices look like second period prices, if you ingnore the part multiplied by $\beta$. We'' look into comparative statics when the prices equations are confimerd to be true.
\section{Perceived [0,y] linear city}
Assume we have two producers on the market: one is incumbent and one is entrant. Suppose location of incumbent and entrant is set: 0 for incumbent and 1 for entrant, however, the consumer perceives the location of entrant as $y>1$ in the first period. The strategies of the producers are then prices: $p_{inc}$ and $p_{e}$ - price of incumbent and entrant respectively. Consumer chooses between the option offered by the incumbent and the entrant (and an outside option, whose value we set to 0), and if in the first period the entrant is chosen, his true location is revealed to the consumer in the second period.

As the original model talks about a set of consumers, whose tastes are distributed uniformly on [0,1], our model's one consumer setting leads to a different demand estimation technique. If v is consumer's valuation of the object and x is his location, which is a random draw from U[0,1], then his utility is:

    $$ U = \begin{cases}
    v - t*x - p_{inc} &  \text{if } v-t*x - p_{inc} \geq v - t(y-x) - p_{e}\\
    v - t*x - p_{e} &  \text{if } v-t*x - p_{inc} < v - t(y-x) - p_{e}
    \end{cases}
$$

Then, demand for incumbent's good is:

$$d_{inc} = \begin{cases}
    1 &  \text{if } x \leq \frac{p_{e}-p_{inc}}{2t} + 1/2y\\
    0 &  \text{if } x > \frac{p_{e}-p_{inc}}{2t} + 1/2y
    \end{cases}
$$
 and vice versus for demand of the entrant.
 
In a one period game incumbent then solves:
$$\max_{p^{inc}} (p^{inc}-c)* Pr(x \leq \frac{p_{e}-p_{inc}}{2t} + 1/2y),$$
which can be simplified to:
$$\max_{p^{inc}} (p^{inc}-c)(\frac{p_{e}-p_{inc}}{2t} + 1/2y)$$
and entrant's profit-maximising problem to:
$$\max_{p^{e}} (p^{e}-c)(\frac{p_{inc}-p_{e}}{2t} + 1 - 1/2y)$$

\subsection{Random consumer location, no network effects}
    
    Suppose we only have one consumer, who is located differently in each period (random draw from U[0,1]). Once consumer is located s.t. the entrant's good is his best option, in the next periods he perceives the true location of the entrant.
    
    Suppose we have a two-period game. Then from the first period point of view, the probability of the entrant being seen in the second period depends on the probability of him being the best option in the first period. Thus, we can write the optimization problem in the first period for both incumbent and entrant as follows:
    
    $$ \begin{aligned}
    \max_{p^{inc}_1, p^{inc}_2} {}
    & \quad (p^{inc}_1-c)* Pr(x_1 \leq \frac{p^{e}_1-p^{inc}_1}{2t} + 1/2y) + \\
    & \quad + \beta((p^{inc}_2-c)(Pr(x_1 \leq \frac{p^{e}_1-p^{inc}_1}{2t} + 1/2y)(Pr(x_2 \leq \frac{p^{e}_2-p^{inc}_2}{2t} + 1/2y)) + \\
    & \quad + (Pr(x_1 > \frac{p^{e}_1-p^{inc}_1}{2t} + 1/2y))(Pr(x_2 \leq \frac{p^{e}_2-p^{inc}_2}{2t} + 1/2)))
    \end{aligned}
    $$

    $$ \begin{aligned}
    \max_{p^{e}_1, p^{e}_2} {}
    & \quad (p^{e}_1-c)* Pr(x_1 > \frac{p^{e}_1-p^{inc}_1}{2t} + 1/2y) + \\
    & \quad + \beta((p^{e}_2-c)(Pr(x_1 \leq \frac{p^{e}_1-p^{inc}_1}{2t} + 1/2y)(Pr(x_2 \leq \frac{p^{inc}_2-p^{e}_2}{2t} + 1 - 1/2y)) + \\
    & \quad + (Pr(x_1 > \frac{p^{e}_1-p^{inc}_1}{2t} + 1/2y))(Pr(x_2 \leq \frac{p^{inc}_2-p^{e}_2}{2t} + 1/2))),
    \end{aligned}
    $$
where $x_1$, $x_2$ are locations of the consumer in each period. 

    
Solving by backward induction, we first discuss the equilibrium conditions in the second period. 
If entrant is noticed, the solution corresponds to classic Hotelling and the profit is then standard $\pi_{2}^{inc} = \pi_{2}^{e}= t/2$. However, if entrant is not noticed, equilibrium prices are as follows:
$$
p^{inc}_2 =1/3t(y+ 2) + c $$
$$ p^{e}_2 =4/3t - 1/3yt + c$$
To make sure that the profits don't go into the negative values, $ p^{e}_2$ should be less than 4, otherwise entrant does not enter the market. Then the profits are $\pi_{2}^{e} = 1/18*t(4-y)^2$ and $\pi_{2}^{inc}=1/18*t(2+y)^2$, respectively. The profit of incumbent is increasing in y and the profit of entrant is decreasing in y for all the allowed values of y ($1<y\leq4$), exponentially. Also notice that if y=1, the profits equations collapse to the classic Hotelling profits. 

Now each producer optimizes their behaviour in the first period, assuming that in each state (multiplied by corresponding probability of reaching it) in the second period they will behave optimally. 

	


 $$
    \max_{p^{inc}_1} {}
   (p^{inc}_1-c)(\frac{p^{e}_1-p^{inc}_1}{2t} + 1/2y) 
    + \beta((\frac{p^{e}_1-p^{inc}_1}{2t} + 1/2y)(\frac{1}{18}t(2+y)^2) + 
    1/2t(\frac{p^{inc}_1-p^{e}_1}{2t} + 1 - 1/2y))
    $$
    
    $$
   \max_{p^{e}_1} {}
   (p^{e}_1-c)(\frac{p^{inc}_1-p^{e}_1}{2t} + 1 -1/2y) 
    + \beta((\frac{p^{e}_1-p^{inc}_1}{2t} + 1/2y)(\frac{1}{18}t(4-y)^2) + 
    1/2t(\frac{p^{inc}_1-p^{e}_1}{2t} + 1 - 1/2y))
    $$

The equilibrium prices in the first period are then:
$$
p^{inc}_1 =1/3t(y+ 2) + c - \frac{\beta t }{54}(y-1)(17+y) $$
$$ p^{e}_1 =4/3t - 1/3yt + c - \frac{\beta t}{54}(y-1)(19-y)$$
Notice that for both the prices the RHS until the $ \beta$ component correspond to our second period equilibrium, the 'unnoticed entrant' state, which in turn collapses to the standard result, when y=1. There is also no dynamic effect in the price, if y=1, as the component becomes 0 - also quite logical, as the dynamic effect appears because of the difference between two states' profits in the second period. 

From the graph it is obvious that high learning cost, modelled via y, has an obvious negative impact on entrant's profit and price. Interestingly, higher valuation of the future profits results in the increase of them for both producers, which makes sense as the price is chosen in a dynamic way so as to receive optimal profits in the secon period as well. The myopic producer does not perceive that there are profits to be had in the second period at all. The most curious is the optimal price of the producer in the non-myopic case: it stays almost the same and even decreases with higher values of y if producer is maximally forward-looking. The incumbent does not dare increase the price in the first period as he is afraid of entrant being noticed; moreso when his loss from entrant being tested is the highest with y the highest. The entrant is also forced to lower his price more and more as compared to the myopic case, as he has a lot to gain from being tested, going for high values of y to 0. The profits of incumbent are increasing exponentially with y regardless, as he is reaping benefits from high probability of him being the closest in consumer's eyes. 

\begin{center}
    \includegraphics[scale=0.7]{random_consumer_location.png}
\end{center}


\includegraphics[scale=0.7]{random_consumer_location_prices.png}
\includegraphics[scale=0.7]{random_consumer_location_profits.png}

\subsection{Random consumer location, network effects}

(no results yet)








%\bibliography{literature.bib} % Filename of bibliography
%\bibliographystyle{apalike}
    
\end{document}
